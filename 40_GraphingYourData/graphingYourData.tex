\documentclass[12pt]{article}

\newcommand{\EventTitle}{Foundation Year, Coding in R}
\newcommand{\PresentationTitle}{7. Graphing your data}
\newcommand{\EventDate}{May 2020}

%\documentclass[12pt]{article}

\usepackage{rotating}

\usepackage{hyperref}

\usepackage{float}
\usepackage[svgnames]{xcolor}



\usepackage{listings}
\lstset{language=R,
    basicstyle=\large\ttfamily,
    stringstyle=\color{DarkGreen},
    otherkeywords={0,1,2,3,4,5,6,7,8,9},
    morekeywords={TRUE,FALSE},
    deletekeywords={data,frame,length,as,character},
    keywordstyle=\color{blue},
    commentstyle=\color{DarkGreen},
    frame=shadowbox,
    rulesepcolor=\color{lightgray},
    backgroundcolor=\color{myyellow},
}
\usepackage[margin=0.3in]{geometry}

\usepackage{xcolor}
\definecolor{red}{rgb}{153,0,153}
\definecolor{blue}{rgb}{0,0,153}
\definecolor{pink}{RGB}{239,231,231}
\definecolor{myyellow}{RGB}{255,255,204}




\usepackage{subfig}
\usepackage{etex}
\reserveinserts{18}
%usepackage{morefloats}



\usepackage{adjustbox} % Used to constrain images to a maximum size 

\usepackage{tikz}
\usetikzlibrary{plotmarks}
\usetikzlibrary{shapes,arrows}
\usetikzlibrary{calc}
\usetikzlibrary{positioning}



\definecolor{MauveDGB}{rgb}{112,108,176}
\definecolor{DGBred}{rgb}{112,108,176}
\usepackage{xspace,colortbl}



\usepackage[official]{eurosym}

\DeclareGraphicsRule{*}{mps}{*}{}

\usepackage{tabularx}

\usetikzlibrary[topaths]





\newcount\mycount




% ------------
\usepackage{helvet}
\usepackage{xspace,colortbl}


\usepackage{graphicx}

\usepackage{xcolor}
\include{imagesEtc/rgb}
\usepackage{sectsty}
\usepackage{float}
\DeclareGraphicsRule{*}{mps}{*}{}
\usepackage{tikz}
\usetikzlibrary{backgrounds}
\usepackage{tabularx}

\usepackage{amssymb}
\usepackage{verbatim}
\usepackage{amssymb}
\usepackage{amsmath}
\usepackage{fancybox}


\usepackage[screen,code,sectionbreak]{pdfscreen}
\begin{screen}
	\margins{.65in}{.65in}{.65in}{.65in}
	\screensize{6.25in}{10in}
	%\changeoverlay
	%\paneloverlay{aquagraphite.jpg}
	%\overlay{mac3.pdf}
	\def\pfill{\vskip6pt}
\end{screen}

\begin{print}
	\setlength{\oddsidemargin}{0in}
	\setlength{\textwidth}{7in}
	\setlength{\topmargin}{-.5in}
	\setlength{\textheight}{9in}
\end{print}


% ------------------------------
% Alter some LaTeX defaults for better treatment of figures:
% See p.105 of "TeX Unbound" for suggested values.
% See pp. 199-200 of Lamport's "LaTeX" book for details.
%   General parameters, for ALL pages:
\renewcommand{\topfraction}{0.9}	% max fraction of floats at top
\renewcommand{\bottomfraction}{0.8}	% max fraction of floats at bottom
%   Parameters for TEXT pages (not float pages):
\setcounter{topnumber}{2}
\setcounter{bottomnumber}{2}
\setcounter{totalnumber}{4}     % 2 may work better
\setcounter{dbltopnumber}{2}    % for 2-column pages
\renewcommand{\dbltopfraction}{0.9}	% fit big float above 2-col. text
\renewcommand{\textfraction}{0.07}	% allow minimal text w. figs
%   Parameters for FLOAT pages (not text pages):
\renewcommand{\floatpagefraction}{0.7}	% require fuller float pages
% N.B.: floatpagefraction MUST be less than topfraction !!
\renewcommand{\dblfloatpagefraction}{0.7}	% require fuller float pages

% remember to use [htp] or [htpb] for placement

% ------------------------------



% ------------------------------
\begin{document}
	\fontfamily{phv}\selectfont
	\overlay{imagesEtc/stripes} 
	\paneloverlay{mac4} \paneloverlay{aquagraphite}
	
	\fontfamily{phv}\selectfont
	%\overlay{stripes} 
	\paneloverlay{mac4} \paneloverlay{aquagraphite}
	
	\definecolor{BlueGreen}{rgb}{0, 0.3686, 0.4314}
	\definecolor{Burgundy}{RGB}{112,108,176}
	\definecolor{olive}{rgb}{0.4118, 0.5725, 0.2275}
	\definecolor{red}{RGB}{71,170,156}
	\definecolor{DGBred}{RGB}{71,170,156}
	
	\definecolor{section0}{RGB}{112,108,176}
	\definecolor{section1}{RGB}{112,108,176}    
	\definecolor{section2}{RGB}{71,170,156}
	\definecolor{section3}{rgb}{.000,.488,.278}
	\definecolor{section4}{rgb}{.000,.371,.000}
	\definecolor{section5}{rgb}{.000,.212,.000}
	
	\definecolor{scarlet}{rgb}{255,0,0}
	
	\sectionfont{\color{red}}
	\subsectionfont{\color{Burgundy}}
	\subsubsectionfont{\color{blue!50!white}}
	
	
	\begin{screen}
		\begin{titlepage}
			\definecolor{BlueGreen}{rgb}{0, 0.3686, 0.4314}
			\definecolor{Burgundy}{RGB}{112,108,176}
			
			
			%\tikz [remember picture,overlay]
			%  \node [yshift=0.06\paperheight,xshift=0.185\paperwidth,inner sep=0pt] at (current page.south west)
			%{\includegraphics[width=0.35\paperwidth,height=0.1\paperheight]{boelogo.png}};
			
			\tikz [remember picture,overlay]
			\node [yshift=0.5\paperheight,xshift=0.26\paperwidth,inner sep=0pt] at (current page.south west){\begin{minipage}{0.4\paperwidth}\raggedright    \textcolor{red}{\textbf {\EventTitle}} \end{minipage}};
			
			
			
			\tikz [remember picture,overlay]
			\node [yshift=-0.64\paperheight,xshift=0.26\paperwidth,inner sep=0pt] at (current page.north west){\begin{minipage}{0.4\paperwidth}\raggedright       \LARGE \textcolor{Burgundy}{\fontsize{26}{20}\selectfont \PresentationTitle} \end{minipage}};
			
			
			\tikz [remember picture,overlay]
			\node [xshift=-0.21\paperwidth, yshift=-0.21\paperheight, inner sep=0pt] at  (current page.north)
			{\includegraphics[width=0.5\paperwidth,height=0.42\paperheight]{imagesEtc/CCBS_greenGlass.jpg}};
			
			\tikz [remember picture,overlay]
			\node [yshift=-0.42\paperheight,xshift=0.21\paperwidth,inner sep=0pt] at (current page.north west)
			{\includegraphics[height=0.1\paperheight,width=0.3\paperwidth]{imagesEtc/BankLogoGrey.pdf}};
			
			
			
			%\tikz [remember picture,overlay]
			%\draw (-0.75\paperwidth,-0.5\paperheight) -- (.5\paperwidth,-0.0\paperheight)}
			
			\tikz [remember picture,overlay]
			{\draw[line width = 0.9mm,color=red] (0.442\paperwidth,-0.23\paperheight) -- (0.842\paperwidth,-0.23\paperheight)}
			
			
			
			
			
			%Date
			%    \tikz [remember picture,overlay]
			%  \node [yshift=-0.55\paperheight,xshift=-0.38\paperwidth,inner sep=0pt] at (current page.north east){\Large \textcolor{red}{\begin{tabular}{l}
			%     \textbf{Date} \\
			%      31 May 2016
			%   \end{tabular}}};
			
			% -------------------------------- Date/Author etc
			
			\bgroup
			\def\arraystretch{1.5}
			\tikz [remember picture,overlay]
			\node [yshift=-0.55\paperheight,xshift=-0.35\paperwidth,inner sep=0pt] at (current page.north east){\begin{minipage}[c]{0.2\paperwidth} \textcolor{red}{\begin{tabular}{l}
							\textbf{\textcolor{Burgundy}{Date}} \\
							\EventDate
			\end{tabular}}\end{minipage}};
			\egroup 
			
			
			%Separating line 2
			\tikz [remember picture,overlay]
			{\draw[color=red] (0.442\paperwidth,-0.27\paperheight) -- (0.842\paperwidth,-0.27\paperheight)}
			
			%Presenter name
			\bgroup
			\def\arraystretch{1.5}
			\tikz [remember picture,overlay]
			\node [yshift=-0.64\paperheight,xshift=-0.35\paperwidth,inner sep=0pt] at (current page.north east){\begin{minipage}[c][0.1in]{0.2\paperwidth} \textcolor{red}{\begin{tabular}{l}
							\textbf{\textcolor{Burgundy}{Author}} \\
							David Barr
			\end{tabular}}\end{minipage}};
			\egroup
			
			%
			% %   \tikz [remember picture,overlay]
			%   \node [yshift=-0.64\paperheight,xshift=-0.35\paperwidth,inner sep=0pt] at (current page.north east){\begin{minipage}[c][0.1in]{0.2\paperwidth} \textcolor{red}{
			%       \bf {Author}
			%      \noindent David Barr
			%   }\end{minipage}};
			
			
			
			
			%Separating line 3
			\tikz [remember picture,overlay]
			{\draw[color=red] (0.442\paperwidth,-0.30\paperheight) -- (0.842\paperwidth,-0.30\paperheight)}
			
			%Presenter email
			\tikz [remember picture,overlay]
			\node [yshift=-0.71\paperheight,xshift=-0.35\paperwidth,inner sep=0pt] at (current page.north east){\begin{minipage}[c][0.1in]{0.2\paperwidth} \textcolor{red}{\begin{tabular}{l}
							david.barr@bankofengland.co.uk
			\end{tabular}}\end{minipage}};
			
			%Disclaimer
			%    \tikz [remember picture,overlay]
			%   \node [yshift=0.06\paperheight,xshift=-0.27\textwidth,inner sep=0pt] at (current page.south east){\begin{minipage}{0.43\paperwidth}\raggedright  %\scriptsize The Bank of England does not accept any liability for misleading or
			%        inaccurate information or omissions in the information provided.\end{minipage}};
			
		\end{titlepage}
	\end{screen}
	
	\begin{screen}
		\overlay{imagesEtc/stripes} %\paneloverlay{mac4} %\paneloverlay{aquagraphite}
		\LARGE
		\bf                                                                                                                                                                           
	\end{screen}                                                                                                                                                                  
	
	
	






\newpage
	
	\setcounter{tocdepth}{2}
	\tableofcontents
	\newpage
\LARGE




\section{Setting up.}
\begin{itemize}
	\item We shall work throught the file \textcolor{red}{FY\_MPol\_graphData.R} which you should copy to your /courseCode folder.
	\begin{itemize}
		\item This will appear as \textcolor{red}{ImportingData\_CSVfiles} in  /courseCode i.e. Windows doesn't show the `.R' extension.
		\item Check in the `Type' column of the folders file list to be sure you have the correct file - it should read `R file'.
	\end{itemize}
	\item Copy the file from your  /courseCode to folder  /myCodeAndData and rename it `myFY\_MPol\_graphData.R'.
\newpage
	\item The data we will work with are in the file \textcolor{red}{FY\_MPol\_Data.csv} which you should copy into your /courseData folder where it  will appear as \textcolor{red}{FY\_MPol\_Data} i.e. Windows will not show the .csv extension. Copy this file to  /myCodeAndData (no need to rename this one).
	\newpage
\item Then...
\begin{enumerate}
	\item Start R studio.
	\item Set your working directory with
	\begin{lstlisting}
setwd("C:/Users/[your number]/Documents/FY_Rcourse/myCodeAndData")
\end{lstlisting}
as explained in the `5. Setting your working directory' slides.
\item Either:
\begin{enumerate}
	\item Create a new file in R Studio (call it whatever you like) then type (or copy and paste if you're feeling lazy)  the lines from these slides 1 at a time as you read through the text. Or...
	\newpage
	\item Open your file `myFY\_MPol\_Data' (File $>$ Open File) and execute the lines 1 by 1 (with ctrl + enter) as you read through the text. 
\end{enumerate}
\end{enumerate}
\end{itemize}


\section{Installing some packages.}
\label{section:installGgplot}
\begin{itemize}
	\item The basic version of R can do a lot of the things that we would want to do in the Bank but there are some add-ons that can do these things better. Something called `ggplot' can produce much better graphs than the basic R can.
	\item We have to install these add-ons, called `packages'. We do this just once and they go into the computer's memory just as R Studio does.
	\item The instruction to install ggplot is:
\begin{lstlisting}
install.packages("ggplot2'')
\end{lstlisting}
	\item Copy this instruction and paste it next to the blue \textcolor{blue}{$>$}  in the Console  of R Studio and hit $<$enter$>$.
	\item R will then perform the installation, this may take several minutes and you'll see a lot of text appear intermittently in the console while it is doing it.
	\item When the  \textcolor{blue}{$>$} reappears you can move to the next section of these notes.
\end{itemize}

\section{Start with some notes, then import the data.}
\begin{itemize}
		\item First we see the introductory notes that identify the program and explain what it does.
\begin{lstlisting}
#
# FY_MPol_graphData.R
#
# Import data from a csv file and 
# use it to draw some graphs with ggplot.
#
# DGB Mar 2020
#
\end{lstlisting}
\newpage
	\item Then we have some more notes, this time telling us about the data. 
\begin{lstlisting}
#  We import the following US data series
#  from the file FY_MPol_Data.csv:
#
#  Money supply (narrow) m1  
#  Money supply (broad) m3
#  Prices, p: consumer price index
#  Short rate(1), rs: 3-month T Bill rate
#  Short rate(2), rff: Fed Funds rate
#  Long rates, rl: 10-year interest rate (yield on 10-year gov bonds)
#  Output, y: Industrial production
#  Unemployment, u: Unemployment rate
#
\end{lstlisting}
\newpage
\item Now we have the first line of R code, with a note explaining what it does. While you are learning R it is a good idea to include this line at the start of all your programs.
\begin{lstlisting}
rm(list = ls())  
               # This line clears R's memory of data so
               # that we start from a clean slate for this
               # program with nothing hanging around
               # from any programs we ran earlier.
\end{lstlisting}

	\item Now we import the data as in the previous session. This time we are using genuine data instead of the toy stuff we used before. 
\begin{lstlisting}
#
# Import the data into a data frame, which we will call mpdata.
#

mpdata <- read.csv("FY_MPol_Data.csv", header=TRUE)
\end{lstlisting}

	\item Have a look at what we've imported,  just to check that we have what we expected.
\begin{lstlisting}	
mpdata[1,]  # Print the first line of the data (plus the header)
				  # to the console.
\end{lstlisting}
\lstinputlisting[language=R, backgroundcolor=\color{lightgray}]{firstLineOfData.out}

\newpage
	\item R will print the first 6 lines of the data with the `head()' command...
\begin{lstlisting}
head(mpdata)  # This prints the first 6 lines (plus the header).
\end{lstlisting}
\lstinputlisting[language=R, backgroundcolor=\color{lightgray}]{headOfData.out}

\newpage
	\item Similarly for the final 6 lines using `tail()'
\begin{lstlisting}
tail(mpdata)  # And this prints the final 6 lines (plus the header).
\end{lstlisting}
\lstinputlisting[language=R, backgroundcolor=\color{lightgray}]{tailOfData.out}

\newpage
	\item Now that we have the data we start a new part of the code with a comment about what we are going to do. 
	\begin{lstlisting}
# --------------- Plot (1) p and m1, (2) The 3 interest rates
\end{lstlisting}
\newpage
\item There are several ways to plot data in R. One of the best is by using the package ggplot. In the next line we tell R that we want to use ggplot with this program using the library() command. (If you get a message in the console saying that it can't find ggplot go back to Section \ref{section:installGgplot}.)


\begin{lstlisting} 
# Load ggplot from the library of packages on this laptop. 
library("ggplot2")
\end{lstlisting}

\newpage
	\item Next we have to do something that looks a little odd. R has a built in set of code that handles dates really well but we have to tell it that, for example, `1960-01-01' is actually a date and not just a string of numbers and dashes. We do this with the following line which makes R treat the  `date' column of mpdata as dates. (The data frame, mpdata, will look the same after this line is executed as it did before.)

\begin{lstlisting}	 
# Tell R to treat the `date' column as dates.
mpdata[["date"]] <- as.Date(mpdata[["date"]])  
\end{lstlisting}


\newpage
	\item Now, at last, we tell R to create a graph.
\begin{lstlisting}
ggplot(mpdata, aes(date)) + geom_line(aes(y=m1),colour="red") +
                            geom_line(aes(y=p*10),colour="blue")
\end{lstlisting}
\begin{itemize}
	\item Explaining what all this means would take some time so it's best to just accept that this line does the trick.
	\item The first part tells R to use ggplot, and to get the data from our data frame `mpdata'. 
	\item The `aes(date)' tells it what to put on the horizontal axis i.e. the `date' column from mpdata.
	\item Then we get the code to draw the first line, which has y values equal to m1, and will be drawn in red.
	\newpage
	\item Finally we add a second line, with y values equal to $p \times 10$, written `p*10' (the $\times 10$ is just to scale up the values so that they look better in the graph when plotted next to the m1 values), and this one will be blue.
\end{itemize}	
	\item And this is what we get: (It will appear in the bottom right window of R Studio if all is well.)
	\begin{figure}[H]  % --------------------------------------------------------------------------------------
			\begin{center}
  			\includegraphics[width=0.8\linewidth]{mpdataPlot_1.pdf}
  			\caption{Output from ggplot of m1 and p*10.}
  			\label{fig:mpdataPlot_1}
  			\end{center}
	\end{figure}         % --------------------------------------------------------------------------------------

	\item The next line plots the 3 interest rates in mpdata.
	\begin{lstlisting}
ggplot(mpdata, aes(date)) + geom_line(aes(y=rs),colour="red") + 
                            geom_line(aes(y=rl),colour="blue") +
                            geom_line(aes(y=rff),colour="green")
\end{lstlisting}
\begin{figure}[H]  % --------------------------------------------------------------------------------------
			\begin{center}
  			\includegraphics[width=0.8\linewidth]{mpdataPlot_2.pdf}
  			\caption{Output from ggplot of 3 interest rates.}
  			\label{fig:mpdataPlot_1}
  			\end{center}
	\end{figure}         % --------------------------------------------------------------------------------------

	\item Finally, we take control of the start and end dates of the graph.
\begin{lstlisting}	
# Plot a sub period 
# (see https://www.neonscience.org/dc-time-series-plot-ggplot-r):

startPlot <- as.Date("2011-01-01")
endPlot <- as.Date("2019-01-01")

plotDates <- c(startPlot, endPlot)
plotDates

\end{lstlisting}
\begin{itemize}
	\item The first two lines create 2 variables, which we choose to call startPlot and endPlot: they take a string of characters the {\em we} recognise as a date (2011-01-01) and tell R to treat it as a date too (just as we did earlier for the `date' column in mpdata).
	\item We then combine these in a single variable called plotDates (our choice of name again) - ggplot has to be given both dates inside a single variable (it's just one of ggplot's rules).
\end{itemize}
\item Now we can get ggplot going again but this time we add\\  `(scale\_x\_date(limits=plotDates))' to give it the info about the start and end dates.
\begin{lstlisting}
ggplot(mpdata, aes(date)) + geom_line(aes(y=m1),colour="red") + 
                          geom_line(aes(y=p*10),colour="blue") + 
                          (scale_x_date(limits=plotDates))
\end{lstlisting}

\begin{figure}[H]  % --------------------------------------------------------------------------------------
			\begin{center}
  			\includegraphics[width=0.8\linewidth]{mpdataPlot_3.pdf}
  			\caption{Short sample output from ggplot of m1 and p*10.}
  			\label{fig:mpdataPlot_1}
  			\end{center}
\end{figure}         % --------------------------------------------------------------------------------------
\end{itemize}

\section{Finding your plots.}
\begin{itemize}
	\item You should now have 3 plots but you can only see one of the.
	\item To see the others click on the back and forward arrows on the `Plots' pane.
	\begin{figure}[H]  % --------------------------------------------------------------------------------------
			\begin{center}
  			\includegraphics[width=0.8\linewidth]{ScreenShots/findOtherPlots.png}
  			\caption{Moving between plots.}
  			\label{fig:movingBetweenPlots}
  			\end{center}
\end{figure}  
\end{itemize}
\begin{center}That's all for now!\end{center}
\end{document}


\end{document}
	
\begin{lstlisting}
#
# importDataFromCSV.R
# 
# Import (or 'load') data from a CSV file.
#
# DGB Mar 2020
#
\end{lstlisting}

\begin{lstlisting}
#   Load data from csv with the top row of the csv used as column headers.
monPolData <- read.csv("monPol_1.csv",header=TRUE)
\end{lstlisting}


	\item And this is what we get in the console:
	\lstinputlisting[language=R]{monPolData.out}


\section{A note on headers}
\end{document}





# Print just 1 column
monPolData$bank

# Print just 1 row (you'll get the header too)

monPolData[1,]
monPolData[2,]

# Print just one element 
monPolData[2,3]

# How different do things look if we don't make the first line of the CSV file into a header in R?
----------
\begin{lstlisting}
#   Load data from csv with the top row of the csv used as the first row of the columns with column headers added by R (as V#).
monPolData_noHeader <- read.csv("monPol_1.csv",header=FALSE)
\end{lstlisting}
monPolData

monPolData_noHeader


# --------------- Load larger data set (from Radziwill book)

# Load data from csv with variable names as headers from the top row of the csv.
mnmData <- read.csv("mnm-clean.csv",header=TRUE)
mnmData_noHeader <- read.csv("mnm-clean.csv",header=FALSE)

# Look at the data...

#       The whole dataset
mnmData

#       Just the first 6 rows
head(mnmData)

#       What difference does including a 'header' make?
head(mnmData_noHeader)


# Print one of the columns
mnmData$student




\section{Earlier text}


\begin{enumerate}
		\item Import data from a spreadsheet or a `Comma Separated Values' (CSV) file :
		\begin{enumerate}
			\item See the \href{https://www.datacamp.com/community/tutorials/r-data-import-tutorial#csv}{Datacamp site} for a comprehensive guide.
			\item Spreadsheet \textcolor{red}{[mnm-clean.xlsx]}: Export the data as a CSV file \textcolor{red}{[mnm-clean.csv]}.  (File$$>$$Export$$>$$Change File Type$$>$$CSV (Comma delimited))
			\item CSV:  \textcolor{red}{[importDataFromCSV.R]}
		\end{enumerate}
		\begin{figure}[H]
			\begin{center}
  			\includegraphics[width=0.8\linewidth]{ScreenShots/Excel_mnmData.png}
  			\caption{Data in Excel.}
  			\label{fig:dataExcel}
  			\end{center}
		\end{figure}
		\begin{figure}[H]
			\begin{center}
  			\includegraphics[width=0.8\linewidth]{ScreenShots/R_mnmData.png}
  			\caption{Data in R.}
  			\label{fig:dataR}
  			\end{center}
		\end{figure}
		\item Data imported in this way are stored in R in something called a `data frame'; more on what this means below.
\end{enumerate}



\end{document}