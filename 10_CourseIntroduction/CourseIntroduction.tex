\documentclass[12pt]{article}

\newcommand{\EventTitle}{Foundation Year, Coding in R}
\newcommand{\PresentationTitle}{1. Introduction}
\newcommand{\EventDate}{May 2020}

%\documentclass[12pt]{article}

\usepackage{rotating}

\usepackage{hyperref}

\usepackage{float}
\usepackage[svgnames]{xcolor}



\usepackage{listings}
\lstset{language=R,
    basicstyle=\large\ttfamily,
    stringstyle=\color{DarkGreen},
    otherkeywords={0,1,2,3,4,5,6,7,8,9},
    morekeywords={TRUE,FALSE},
    deletekeywords={data,frame,length,as,character},
    keywordstyle=\color{blue},
    commentstyle=\color{DarkGreen},
    frame=shadowbox,
    rulesepcolor=\color{lightgray},
    backgroundcolor=\color{myyellow},
}
\usepackage[margin=0.3in]{geometry}

\usepackage{xcolor}
\definecolor{red}{rgb}{153,0,153}
\definecolor{blue}{rgb}{0,0,153}
\definecolor{pink}{RGB}{239,231,231}
\definecolor{myyellow}{RGB}{255,255,204}




\usepackage{subfig}
\usepackage{etex}
\reserveinserts{18}
%usepackage{morefloats}



\usepackage{adjustbox} % Used to constrain images to a maximum size 

\usepackage{tikz}
\usetikzlibrary{plotmarks}
\usetikzlibrary{shapes,arrows}
\usetikzlibrary{calc}
\usetikzlibrary{positioning}



\definecolor{MauveDGB}{rgb}{112,108,176}
\definecolor{DGBred}{rgb}{112,108,176}
\usepackage{xspace,colortbl}



\usepackage[official]{eurosym}

\DeclareGraphicsRule{*}{mps}{*}{}

\usepackage{tabularx}

\usetikzlibrary[topaths]





\newcount\mycount




% ------------
\usepackage{helvet}
\usepackage{xspace,colortbl}


\usepackage{graphicx}

\usepackage{xcolor}
\include{imagesEtc/rgb}
\usepackage{sectsty}
\usepackage{float}
\DeclareGraphicsRule{*}{mps}{*}{}
\usepackage{tikz}
\usetikzlibrary{backgrounds}
\usepackage{tabularx}

\usepackage{amssymb}
\usepackage{verbatim}
\usepackage{amssymb}
\usepackage{amsmath}
\usepackage{fancybox}


\usepackage[screen,code,sectionbreak]{pdfscreen}
\begin{screen}
	\margins{.65in}{.65in}{.65in}{.65in}
	\screensize{6.25in}{10in}
	%\changeoverlay
	%\paneloverlay{aquagraphite.jpg}
	%\overlay{mac3.pdf}
	\def\pfill{\vskip6pt}
\end{screen}

\begin{print}
	\setlength{\oddsidemargin}{0in}
	\setlength{\textwidth}{7in}
	\setlength{\topmargin}{-.5in}
	\setlength{\textheight}{9in}
\end{print}


% ------------------------------
% Alter some LaTeX defaults for better treatment of figures:
% See p.105 of "TeX Unbound" for suggested values.
% See pp. 199-200 of Lamport's "LaTeX" book for details.
%   General parameters, for ALL pages:
\renewcommand{\topfraction}{0.9}	% max fraction of floats at top
\renewcommand{\bottomfraction}{0.8}	% max fraction of floats at bottom
%   Parameters for TEXT pages (not float pages):
\setcounter{topnumber}{2}
\setcounter{bottomnumber}{2}
\setcounter{totalnumber}{4}     % 2 may work better
\setcounter{dbltopnumber}{2}    % for 2-column pages
\renewcommand{\dbltopfraction}{0.9}	% fit big float above 2-col. text
\renewcommand{\textfraction}{0.07}	% allow minimal text w. figs
%   Parameters for FLOAT pages (not text pages):
\renewcommand{\floatpagefraction}{0.7}	% require fuller float pages
% N.B.: floatpagefraction MUST be less than topfraction !!
\renewcommand{\dblfloatpagefraction}{0.7}	% require fuller float pages

% remember to use [htp] or [htpb] for placement

% ------------------------------



% ------------------------------
\begin{document}
	\fontfamily{phv}\selectfont
	\overlay{imagesEtc/stripes} 
	\paneloverlay{mac4} \paneloverlay{aquagraphite}
	
	\fontfamily{phv}\selectfont
	%\overlay{stripes} 
	\paneloverlay{mac4} \paneloverlay{aquagraphite}
	
	\definecolor{BlueGreen}{rgb}{0, 0.3686, 0.4314}
	\definecolor{Burgundy}{RGB}{112,108,176}
	\definecolor{olive}{rgb}{0.4118, 0.5725, 0.2275}
	\definecolor{red}{RGB}{71,170,156}
	\definecolor{DGBred}{RGB}{71,170,156}
	
	\definecolor{section0}{RGB}{112,108,176}
	\definecolor{section1}{RGB}{112,108,176}    
	\definecolor{section2}{RGB}{71,170,156}
	\definecolor{section3}{rgb}{.000,.488,.278}
	\definecolor{section4}{rgb}{.000,.371,.000}
	\definecolor{section5}{rgb}{.000,.212,.000}
	
	\definecolor{scarlet}{rgb}{255,0,0}
	
	\sectionfont{\color{red}}
	\subsectionfont{\color{Burgundy}}
	\subsubsectionfont{\color{blue!50!white}}
	
	
	\begin{screen}
		\begin{titlepage}
			\definecolor{BlueGreen}{rgb}{0, 0.3686, 0.4314}
			\definecolor{Burgundy}{RGB}{112,108,176}
			
			
			%\tikz [remember picture,overlay]
			%  \node [yshift=0.06\paperheight,xshift=0.185\paperwidth,inner sep=0pt] at (current page.south west)
			%{\includegraphics[width=0.35\paperwidth,height=0.1\paperheight]{boelogo.png}};
			
			\tikz [remember picture,overlay]
			\node [yshift=0.5\paperheight,xshift=0.26\paperwidth,inner sep=0pt] at (current page.south west){\begin{minipage}{0.4\paperwidth}\raggedright    \textcolor{red}{\textbf {\EventTitle}} \end{minipage}};
			
			
			
			\tikz [remember picture,overlay]
			\node [yshift=-0.64\paperheight,xshift=0.26\paperwidth,inner sep=0pt] at (current page.north west){\begin{minipage}{0.4\paperwidth}\raggedright       \LARGE \textcolor{Burgundy}{\fontsize{26}{20}\selectfont \PresentationTitle} \end{minipage}};
			
			
			\tikz [remember picture,overlay]
			\node [xshift=-0.21\paperwidth, yshift=-0.21\paperheight, inner sep=0pt] at  (current page.north)
			{\includegraphics[width=0.5\paperwidth,height=0.42\paperheight]{imagesEtc/CCBS_greenGlass.jpg}};
			
			\tikz [remember picture,overlay]
			\node [yshift=-0.42\paperheight,xshift=0.21\paperwidth,inner sep=0pt] at (current page.north west)
			{\includegraphics[height=0.1\paperheight,width=0.3\paperwidth]{imagesEtc/BankLogoGrey.pdf}};
			
			
			
			%\tikz [remember picture,overlay]
			%\draw (-0.75\paperwidth,-0.5\paperheight) -- (.5\paperwidth,-0.0\paperheight)}
			
			\tikz [remember picture,overlay]
			{\draw[line width = 0.9mm,color=red] (0.442\paperwidth,-0.23\paperheight) -- (0.842\paperwidth,-0.23\paperheight)}
			
			
			
			
			
			%Date
			%    \tikz [remember picture,overlay]
			%  \node [yshift=-0.55\paperheight,xshift=-0.38\paperwidth,inner sep=0pt] at (current page.north east){\Large \textcolor{red}{\begin{tabular}{l}
			%     \textbf{Date} \\
			%      31 May 2016
			%   \end{tabular}}};
			
			% -------------------------------- Date/Author etc
			
			\bgroup
			\def\arraystretch{1.5}
			\tikz [remember picture,overlay]
			\node [yshift=-0.55\paperheight,xshift=-0.35\paperwidth,inner sep=0pt] at (current page.north east){\begin{minipage}[c]{0.2\paperwidth} \textcolor{red}{\begin{tabular}{l}
							\textbf{\textcolor{Burgundy}{Date}} \\
							\EventDate
			\end{tabular}}\end{minipage}};
			\egroup 
			
			
			%Separating line 2
			\tikz [remember picture,overlay]
			{\draw[color=red] (0.442\paperwidth,-0.27\paperheight) -- (0.842\paperwidth,-0.27\paperheight)}
			
			%Presenter name
			\bgroup
			\def\arraystretch{1.5}
			\tikz [remember picture,overlay]
			\node [yshift=-0.64\paperheight,xshift=-0.35\paperwidth,inner sep=0pt] at (current page.north east){\begin{minipage}[c][0.1in]{0.2\paperwidth} \textcolor{red}{\begin{tabular}{l}
							\textbf{\textcolor{Burgundy}{Author}} \\
							David Barr
			\end{tabular}}\end{minipage}};
			\egroup
			
			%
			% %   \tikz [remember picture,overlay]
			%   \node [yshift=-0.64\paperheight,xshift=-0.35\paperwidth,inner sep=0pt] at (current page.north east){\begin{minipage}[c][0.1in]{0.2\paperwidth} \textcolor{red}{
			%       \bf {Author}
			%      \noindent David Barr
			%   }\end{minipage}};
			
			
			
			
			%Separating line 3
			\tikz [remember picture,overlay]
			{\draw[color=red] (0.442\paperwidth,-0.30\paperheight) -- (0.842\paperwidth,-0.30\paperheight)}
			
			%Presenter email
			\tikz [remember picture,overlay]
			\node [yshift=-0.71\paperheight,xshift=-0.35\paperwidth,inner sep=0pt] at (current page.north east){\begin{minipage}[c][0.1in]{0.2\paperwidth} \textcolor{red}{\begin{tabular}{l}
							david.barr@bankofengland.co.uk
			\end{tabular}}\end{minipage}};
			
			%Disclaimer
			%    \tikz [remember picture,overlay]
			%   \node [yshift=0.06\paperheight,xshift=-0.27\textwidth,inner sep=0pt] at (current page.south east){\begin{minipage}{0.43\paperwidth}\raggedright  %\scriptsize The Bank of England does not accept any liability for misleading or
			%        inaccurate information or omissions in the information provided.\end{minipage}};
			
		\end{titlepage}
	\end{screen}
	
	\begin{screen}
		\overlay{imagesEtc/stripes} %\paneloverlay{mac4} %\paneloverlay{aquagraphite}
		\LARGE
		\bf                                                                                                                                                                           
	\end{screen}                                                                                                                                                                  
	
	
	






\newpage
	
	\setcounter{tocdepth}{2}
	\tableofcontents
	\newpage
\LARGE

\section{Overview.}
\begin{itemize}
	\item This short course, part of the 2020 Foundation Year, is intended to introduce you to coding and, more specifically, to coding in R.
	\item The course is part of a Bank-wide initiative to increase the level of data skills in the Bank; you may have noticed similar, more specialised courses, springing up across the Bank recently.
	\item The session that you had with Louisa in the Microprudential module looked at data skills in the wider context of a competency framework. 
		\begin{figure}[H]  % --------------------------------------------------------------------------------------
			\begin{center}
  			\includegraphics[width=0.7\linewidth]{ScreenShots/CompetencyFrameworkDataAnalytics.png}
  			\caption{Competency framework: Data analytics.}
  			\label{fig:CompFramework}
  			\end{center}
\end{figure}         % --------------------------------------------------------------------------------------
	\item The elements of Louisa's framework that we will be focusing on here come from, mainly,
	\begin{itemize}
		\item Data analysis.
		\item Presenting and sharing data.
	\end{itemize}
	\item We will be working mainly at Level 1 (Foundation) with some elements of Level 2 (Intermediate).
	\item We will also touch on Data preparation at Level 1.
		\begin{figure}[H]  % --------------------------------------------------------------------------------------
			\begin{center}
  			\includegraphics[width=0.7\linewidth]{ScreenShots/CompetencyFrameworkDataAnalytics_H.png}
  			\caption{Competency framework: Broad areas for the Foundation Year.}
  			\label{fig:CompFramework}
  			\end{center}
\end{figure}         % --------------------------------------------------------------------------------------
	\item After the introductory sessions we will use topics from the Monetary Policy, Markets and Banking, and Financial Stability modules as vehicles for introducing aspects of R. These later sessions will, therefore, cover both economics/finance and coding in R.  
\end{itemize}

\section{What if you know nothing about coding, and would like to keep it that way?}
\begin{itemize}
	\item Many people have no interest in coding and even less interest in learning about it.
	\item This course is not intended to turn you into a `geek'! Its  \textcolor{red}{\em most basic objective} is to make you aware of what coding, and R, can achieve so that you have some idea of what someone who can code could provide for you, whether it is to generate a chart, undertake some statistical analysis, use R or some other language to automate some repetitive task, or even solve some mathematical puzzle that you're struggling with.
	\newpage
	\item So you could get something out of the course without writing a single line of code. It is, however, definitely worth having a go at some of the examples in the course - you'll find that coding is a lot easier (and, just maybe, more fun) than you think!
	\item \textcolor{red}{\em We all make mistakes!} It is worth remembering this. Even professional programmers make many mistakes in the course of creating a programme and they (should) always allow time to find and correct them. Coding is not a test, you can't break anything when you make mistakes and you can just delete any mess you make, walk away, pretend it never happened and try again later. You can think of your mistakes as equivalent to the things that you change when redrafting an essay; the only difference being that the computer will let you know when you make one when coding.
	\item You will be a member of a study group of 5 or 6 people, and each group will include at least one person with some coding experience - you should feel free to ask them for advice but bear in mind that they may be working in a pressured area and might not have much time available.
	\item If you get stuck you are welcome to contact me, email is best, and I will get back to you as soon as I can.
\end{itemize}

\section{What if you are already proficient at coding?}
\begin{itemize}
	\item The first few slide sets will not be new to you although it would be useful if you could read through them quickly so that you know what everyone else is doing. 			
	\item The topics that might be new to you are the ones that deal with finance and economics. These will contain non-coding material that you may find interesting, and the coding examples that go with them might contain some new ideas or methods.
	\newpage
	\item Those of you with experience of R will know that there are many ways of achieving even the simplest of tasks. In order to keep things manageable for those who are new to R we typically look at only 1 method for each task. If you are helping others please keep this in mind and try to avoid bombarding them with alternatives even if they are better than those in the slides.   
\end{itemize}

\section{How will the content be delivered?}
\begin{itemize}
	\item In common with the rest of the FY our baseline delivery method will be notes for you to read when you have time to do so. We will look using at more-interactive methods as these become available.
	\item The course will be spread over the remaining modules. Ideally each R session would be linked to the module topic - we aim to achieve this but it will not always be possible.
	\item Most of the session notes will be based on their own R program (these will be available on Moodle, or may be emailed to you) which the notes will explain line by line. All of the code will be highlighted in these notes; to get the most out of them try to type each line into R and run them yourself. 
	\newpage 
	\item You will be able to copy and paste from the notes but you will learn more  by typing the code yourself.
	\item You will probably make mistakes while typing code (e.g. simple typos, using lower case when upper case is needed etc) but this is how we learn; ``Learning by doing it wrong'' is very common in this business.
	\item At the end of some of the session notes there will be a few exercises. Some of these will be memory tests or puzzles that you can do without having R Studio available, others will be coding puzzles for you to solve by writing and running your own code.
	\item {\em The modules' self assessment tests will include a few questions based on R.}
\end{itemize} 

\section{What comes next?}
\begin{enumerate}
	\item We shall start by installing R, along with a popular user interface called R Studio, on your laptop.
	\item Then we set up some folders in which to keep your programmes and data.
	\item At this point we will be ready to run a couple of lines of code in R Studio starting with individual lines in the `console', and then creating a program file to run both lines together.
	\item Next we load some data related to monetary policy.
	\item And then we analyse the data using graphs and some regressions.
	\newpage
	\item After that we mix R with finance and use it to learn about bond yields, financial derivatives, risk management, implied volatilities and Monte Carlo simulation, among other things.
	\item The course is not heavily loaded, particularly for those with some coding experience, so if there are any finance topics that you would like to have included just let me know. I can't guarantee to include them but will defintiely consider it! 
\end{enumerate}


\end{document}