\documentclass[12pt]{article}

\newcommand{\EventTitle}{Foundation Year, Coding in R}
\newcommand{\PresentationTitle}{6. Importing data.}
\newcommand{\EventDate}{May 2020}

%\documentclass[12pt]{article}

\usepackage{rotating}

\usepackage{hyperref}

\usepackage{float}
\usepackage[svgnames]{xcolor}



\usepackage{listings}
\lstset{language=R,
    basicstyle=\large\ttfamily,
    stringstyle=\color{DarkGreen},
    otherkeywords={0,1,2,3,4,5,6,7,8,9},
    morekeywords={TRUE,FALSE},
    deletekeywords={data,frame,length,as,character},
    keywordstyle=\color{blue},
    commentstyle=\color{DarkGreen},
    frame=shadowbox,
    rulesepcolor=\color{lightgray},
    backgroundcolor=\color{myyellow},
}
\usepackage[margin=0.3in]{geometry}

\usepackage{xcolor}
\definecolor{red}{rgb}{153,0,153}
\definecolor{blue}{rgb}{0,0,153}
\definecolor{pink}{RGB}{239,231,231}
\definecolor{myyellow}{RGB}{255,255,204}




\usepackage{subfig}
\usepackage{etex}
\reserveinserts{18}
%usepackage{morefloats}



\usepackage{adjustbox} % Used to constrain images to a maximum size 

\usepackage{tikz}
\usetikzlibrary{plotmarks}
\usetikzlibrary{shapes,arrows}
\usetikzlibrary{calc}
\usetikzlibrary{positioning}



\definecolor{MauveDGB}{rgb}{112,108,176}
\definecolor{DGBred}{rgb}{112,108,176}
\usepackage{xspace,colortbl}



\usepackage[official]{eurosym}

\DeclareGraphicsRule{*}{mps}{*}{}

\usepackage{tabularx}

\usetikzlibrary[topaths]





\newcount\mycount




% ------------
\usepackage{helvet}
\usepackage{xspace,colortbl}


\usepackage{graphicx}

\usepackage{xcolor}
\include{imagesEtc/rgb}
\usepackage{sectsty}
\usepackage{float}
\DeclareGraphicsRule{*}{mps}{*}{}
\usepackage{tikz}
\usetikzlibrary{backgrounds}
\usepackage{tabularx}

\usepackage{amssymb}
\usepackage{verbatim}
\usepackage{amssymb}
\usepackage{amsmath}
\usepackage{fancybox}


\usepackage[screen,code,sectionbreak]{pdfscreen}
\begin{screen}
	\margins{.65in}{.65in}{.65in}{.65in}
	\screensize{6.25in}{10in}
	%\changeoverlay
	%\paneloverlay{aquagraphite.jpg}
	%\overlay{mac3.pdf}
	\def\pfill{\vskip6pt}
\end{screen}

\begin{print}
	\setlength{\oddsidemargin}{0in}
	\setlength{\textwidth}{7in}
	\setlength{\topmargin}{-.5in}
	\setlength{\textheight}{9in}
\end{print}


% ------------------------------
% Alter some LaTeX defaults for better treatment of figures:
% See p.105 of "TeX Unbound" for suggested values.
% See pp. 199-200 of Lamport's "LaTeX" book for details.
%   General parameters, for ALL pages:
\renewcommand{\topfraction}{0.9}	% max fraction of floats at top
\renewcommand{\bottomfraction}{0.8}	% max fraction of floats at bottom
%   Parameters for TEXT pages (not float pages):
\setcounter{topnumber}{2}
\setcounter{bottomnumber}{2}
\setcounter{totalnumber}{4}     % 2 may work better
\setcounter{dbltopnumber}{2}    % for 2-column pages
\renewcommand{\dbltopfraction}{0.9}	% fit big float above 2-col. text
\renewcommand{\textfraction}{0.07}	% allow minimal text w. figs
%   Parameters for FLOAT pages (not text pages):
\renewcommand{\floatpagefraction}{0.7}	% require fuller float pages
% N.B.: floatpagefraction MUST be less than topfraction !!
\renewcommand{\dblfloatpagefraction}{0.7}	% require fuller float pages

% remember to use [htp] or [htpb] for placement

% ------------------------------



% ------------------------------
\begin{document}
	\fontfamily{phv}\selectfont
	\overlay{imagesEtc/stripes} 
	\paneloverlay{mac4} \paneloverlay{aquagraphite}
	
	\fontfamily{phv}\selectfont
	%\overlay{stripes} 
	\paneloverlay{mac4} \paneloverlay{aquagraphite}
	
	\definecolor{BlueGreen}{rgb}{0, 0.3686, 0.4314}
	\definecolor{Burgundy}{RGB}{112,108,176}
	\definecolor{olive}{rgb}{0.4118, 0.5725, 0.2275}
	\definecolor{red}{RGB}{71,170,156}
	\definecolor{DGBred}{RGB}{71,170,156}
	
	\definecolor{section0}{RGB}{112,108,176}
	\definecolor{section1}{RGB}{112,108,176}    
	\definecolor{section2}{RGB}{71,170,156}
	\definecolor{section3}{rgb}{.000,.488,.278}
	\definecolor{section4}{rgb}{.000,.371,.000}
	\definecolor{section5}{rgb}{.000,.212,.000}
	
	\definecolor{scarlet}{rgb}{255,0,0}
	
	\sectionfont{\color{red}}
	\subsectionfont{\color{Burgundy}}
	\subsubsectionfont{\color{blue!50!white}}
	
	
	\begin{screen}
		\begin{titlepage}
			\definecolor{BlueGreen}{rgb}{0, 0.3686, 0.4314}
			\definecolor{Burgundy}{RGB}{112,108,176}
			
			
			%\tikz [remember picture,overlay]
			%  \node [yshift=0.06\paperheight,xshift=0.185\paperwidth,inner sep=0pt] at (current page.south west)
			%{\includegraphics[width=0.35\paperwidth,height=0.1\paperheight]{boelogo.png}};
			
			\tikz [remember picture,overlay]
			\node [yshift=0.5\paperheight,xshift=0.26\paperwidth,inner sep=0pt] at (current page.south west){\begin{minipage}{0.4\paperwidth}\raggedright    \textcolor{red}{\textbf {\EventTitle}} \end{minipage}};
			
			
			
			\tikz [remember picture,overlay]
			\node [yshift=-0.64\paperheight,xshift=0.26\paperwidth,inner sep=0pt] at (current page.north west){\begin{minipage}{0.4\paperwidth}\raggedright       \LARGE \textcolor{Burgundy}{\fontsize{26}{20}\selectfont \PresentationTitle} \end{minipage}};
			
			
			\tikz [remember picture,overlay]
			\node [xshift=-0.21\paperwidth, yshift=-0.21\paperheight, inner sep=0pt] at  (current page.north)
			{\includegraphics[width=0.5\paperwidth,height=0.42\paperheight]{imagesEtc/CCBS_greenGlass.jpg}};
			
			\tikz [remember picture,overlay]
			\node [yshift=-0.42\paperheight,xshift=0.21\paperwidth,inner sep=0pt] at (current page.north west)
			{\includegraphics[height=0.1\paperheight,width=0.3\paperwidth]{imagesEtc/BankLogoGrey.pdf}};
			
			
			
			%\tikz [remember picture,overlay]
			%\draw (-0.75\paperwidth,-0.5\paperheight) -- (.5\paperwidth,-0.0\paperheight)}
			
			\tikz [remember picture,overlay]
			{\draw[line width = 0.9mm,color=red] (0.442\paperwidth,-0.23\paperheight) -- (0.842\paperwidth,-0.23\paperheight)}
			
			
			
			
			
			%Date
			%    \tikz [remember picture,overlay]
			%  \node [yshift=-0.55\paperheight,xshift=-0.38\paperwidth,inner sep=0pt] at (current page.north east){\Large \textcolor{red}{\begin{tabular}{l}
			%     \textbf{Date} \\
			%      31 May 2016
			%   \end{tabular}}};
			
			% -------------------------------- Date/Author etc
			
			\bgroup
			\def\arraystretch{1.5}
			\tikz [remember picture,overlay]
			\node [yshift=-0.55\paperheight,xshift=-0.35\paperwidth,inner sep=0pt] at (current page.north east){\begin{minipage}[c]{0.2\paperwidth} \textcolor{red}{\begin{tabular}{l}
							\textbf{\textcolor{Burgundy}{Date}} \\
							\EventDate
			\end{tabular}}\end{minipage}};
			\egroup 
			
			
			%Separating line 2
			\tikz [remember picture,overlay]
			{\draw[color=red] (0.442\paperwidth,-0.27\paperheight) -- (0.842\paperwidth,-0.27\paperheight)}
			
			%Presenter name
			\bgroup
			\def\arraystretch{1.5}
			\tikz [remember picture,overlay]
			\node [yshift=-0.64\paperheight,xshift=-0.35\paperwidth,inner sep=0pt] at (current page.north east){\begin{minipage}[c][0.1in]{0.2\paperwidth} \textcolor{red}{\begin{tabular}{l}
							\textbf{\textcolor{Burgundy}{Author}} \\
							David Barr
			\end{tabular}}\end{minipage}};
			\egroup
			
			%
			% %   \tikz [remember picture,overlay]
			%   \node [yshift=-0.64\paperheight,xshift=-0.35\paperwidth,inner sep=0pt] at (current page.north east){\begin{minipage}[c][0.1in]{0.2\paperwidth} \textcolor{red}{
			%       \bf {Author}
			%      \noindent David Barr
			%   }\end{minipage}};
			
			
			
			
			%Separating line 3
			\tikz [remember picture,overlay]
			{\draw[color=red] (0.442\paperwidth,-0.30\paperheight) -- (0.842\paperwidth,-0.30\paperheight)}
			
			%Presenter email
			\tikz [remember picture,overlay]
			\node [yshift=-0.71\paperheight,xshift=-0.35\paperwidth,inner sep=0pt] at (current page.north east){\begin{minipage}[c][0.1in]{0.2\paperwidth} \textcolor{red}{\begin{tabular}{l}
							david.barr@bankofengland.co.uk
			\end{tabular}}\end{minipage}};
			
			%Disclaimer
			%    \tikz [remember picture,overlay]
			%   \node [yshift=0.06\paperheight,xshift=-0.27\textwidth,inner sep=0pt] at (current page.south east){\begin{minipage}{0.43\paperwidth}\raggedright  %\scriptsize The Bank of England does not accept any liability for misleading or
			%        inaccurate information or omissions in the information provided.\end{minipage}};
			
		\end{titlepage}
	\end{screen}
	
	\begin{screen}
		\overlay{imagesEtc/stripes} %\paneloverlay{mac4} %\paneloverlay{aquagraphite}
		\LARGE
		\bf                                                                                                                                                                           
	\end{screen}                                                                                                                                                                  
	
	
	






\newpage
	
	\setcounter{tocdepth}{2}
	\tableofcontents
	\newpage
\LARGE

\newpage

\section{Setting up.}
\begin{itemize}
	\item We shall work throught the file \textcolor{red}{ImportingData\_CSVfiles.R} which you should copy to your /courseCode folder.
	\begin{itemize}
		\item This will appear as \textcolor{red}{ImportingData\_CSVfiles} in  /courseCode i.e. Windows doesn't show the `.R' extension.
		\item Check in the `Type' column of the folders file list to be sure you have the correct file - it should read `R file'.
	\end{itemize}
	\item Copy the file from your  /courseCode to folder  /myCodeAndData and rename it `myImportingData\_CSVfiles.R'.
\newpage
	\item The data we will load is in the file \textcolor{red}{monPol\_1.csv} which you should copy into your /courseData folder where it  will appear as \textcolor{red}{monPol\_1} i.e. Windows will not show the .csv extension. Copy this file to  /myCodeAndData (no need to rename this one).
	\newpage
\item Then...
\begin{enumerate}
	\item Start R studio.
	\item Set your working directory with
	\begin{lstlisting}
setwd("C:/Users/[your number]/Documents/FY_Rcourse/myCodeAndData")
\end{lstlisting}
as explained in the `5. Setting your working directory' slides.
\item Either:
\begin{enumerate}
	\item Create a new file in R Studio (call it whatever you like) then type (or copy and paste if you're feeling lazy)  the lines from these slides 1 at a time as you read through the text. Or...
	\newpage
	\item Open your file `myImportingData\_CSVfiles.R' (File $>$ Open File) and execute the lines 1 by 1 (with ctrl + enter) as you read through the text. 
\end{enumerate}
\end{enumerate}
\end{itemize}


\section{Importing the contents of a small CSV file.}

\begin{itemize}
	\item The program starts with some lines telling us its name, what it does, and who wrote it and when.
	\item Lines beginning with a \# are `comment lines' - R ignores these when it executes the code.
	\begin{itemize}
		\item	It's good practice to start your code files this way because this information may be useful to other people or to you when you return to a file that you wrote months ago.
	\end{itemize}
	\newpage
\begin{lstlisting}
#
# importingData_CSVfiles.R
# 
# Import (or `load') data from a CSV file.
#
# DGB Mar 2020
#
\end{lstlisting}

\newpage
	\item Next we get some reminders about how to execute the code. 
	\begin{itemize}
		\item Including additional notes can be useful sometimes. These particular notes are included here only because we are just starting with R.
	\end{itemize}

\begin{lstlisting}
# To execute a single line place the cursor 
#              on the line and hit <ctrl><enter>
# To execute a block of lines, highlight them and hit <ctrl><enter>
# To execute the whole file: <ctrl><shift><enter> 
\end{lstlisting}

	\item Now we have a note to tell us what the code is about to do.
\begin{lstlisting}	
# --------------- Load a small data set:
\end{lstlisting}
\newpage
	\item And then we get to some actual code after all the comment lines.
\begin{lstlisting}
#   Load data from csv with the top row of the csv used 
#           as column headers.
monPolData <- read.csv("monPol_1.csv",header=TRUE)
\end{lstlisting}
\item So what is going on here?
	\begin{itemize}
		\item First there is a note in green, which we'll come back to below.
		\item Then we get the R code in the line 
\begin{lstlisting}
monPolData <- read.csv("monPol_1.csv",header=TRUE)
\end{lstlisting}
		\item We're telling R to read a csv file with the name `monPol\_1.csv' and to place the contents of the file in its memory and call it `monPolData'.
		\newpage
		\item You might be more used to seeing this sort of instruction with an equals sign rather than an arrow.
		\begin{itemize}
			\item Using  `=' would work here too but you'll often see the arrow notation when looking at other people's code which is why we use it here.
			\item What it is doing is telling R to create a space in its memory with the label `monPolData' and to put the data from the csv file into that space.
			\item If we ask R to use this data later will we refer to it as `monPolData' and R knows to go and get the contents of this bit of its memory.
			\item If you don't like the sound of all this, thinking of `monPolData' as the name of the data, or as a variable, works just fine.
	\end{itemize}
\newpage
		\item When we read in a CSV like this the resulting variable has a particular form - it's called a \href{http://www.r-tutor.com/r-introduction/data-frame}{`data frame'}.
		\item Data frames deserve a book to themselves but for now we just note that they are a very convenient way in which to keep our data.
		\item R creates the variable `monPolData' in response to this line of code i.e. we don't have to create it in advance of importing the data.
		\item `monPolData' will live inside R's memory: it will not appear in any of your folders since it is a data frame, not a file.
	\end{itemize}
\newpage
	\item This is what our original csv file looks like in a text editor (Notepad), which makes the `comma separated variable' format obvious. {\em Note the variable names in the first row.}
		\begin{figure}[H]
			\begin{center}
  			\includegraphics[width=0.6\linewidth]{ScreenShots/dataInNotepad.png}
  			\caption{monPol\_1.csv in Notepad.}
  			\label{fig:dataR}
  			\end{center}
		\end{figure}
\newpage
	\item Next we print our data frame in the console.
\begin{lstlisting}
# Print the data in the console.
monPolData
\end{lstlisting}
	\begin{itemize}
		\item To print something from R's memory in the console we don't have to tell R to `print' it, we just write its name on a line {\em on its own}.
     \end{itemize}
\newpage
	\item This is what we get in the console:
	\lstinputlisting[language=R, backgroundcolor=\color{lightgray}]{monPolData.out}
	\begin{itemize}	
		\item This is R's version of our CSV file. It has 4 numbered rows of data plus a header with a name for each of the columns.
		\begin{itemize}
			\item When we loaded the data the `header=TRUE' part told R to treat the first row of the csv file as the column headers.
		\end{itemize}
		\item The column names are useful because they will allow us to pull out individual columns by name, which is what we do next.
	\end{itemize}
\newpage
	\item Print a single column to the console.
\begin{lstlisting}
# Print just 1 column
monPolData$bank
\end{lstlisting}
	\begin{itemize}
		\item The name before the \$ sign tells R which data frame we want to get the column from (we have only one of course); the one after it identifies the column.
		\item As before, we do not have to instruct R to print, just writing the name like this is enough.
\newpage
		\item And this is what we get:
		\lstinputlisting[language=R,backgroundcolor=\color{lightgray}]{monPolData_bank.out}
		\item It doesn't look much like a column, but that's just how columns get printed in the console. The console cannot format what it prints - it just puts eveything in to a single line here.
		\item The data are there though i.e. the `bank' column just contains the names of some countries (referring to the home of the central bank that set the rates in the second column).
		\item You can ignore the line begining with `Levels:'.
	\end{itemize}
\newpage
	\item Creating a variable out of a column.
	\begin{lstlisting}
	# Create a variable out of a column
	r <- monPol$rate
	\end{lstlisting}
	\begin{itemize}
		\item We use the same \$ notation as before but this time we tell R to put the column in a variable called `r'.
		\item We can use this variable in may ways but for now we just print it to the console.
\begin{lstlisting}
# Print the new variable
r
\end{lstlisting}
		 \item And we get:
		 \lstinputlisting[language=R, backgroundcolor=\color{lightgray}]{r.out}
%		\newpage
%		 \item Note that we do not get any header information - the variable contains only the data from the numbered rows of the data frame.
%		 \item This makes it much easier to work with later e.g. if we want to graph the variable we won't want the first observation to be a word (which would be `rate' in this case).
		 \end{itemize}
\newpage
		 \item We can also pull out a single {\em row} to print or to use to create another variable.
		 \item We do this twice in the next example, for rows 1 and 2.
\begin{lstlisting}
# Print just 1 row (you'll get the header too)

monPolData[1,]
monPolData[2,]
\end{lstlisting}
\begin{itemize}
\item This gives us:
\lstinputlisting[language=R, backgroundcolor=\color{lightgray}]{singleRows.out}
\item Note that this is the output from 2 lines of code. Both outputs include the header from the data frame.
\end{itemize}	
	\item When we create a {\em variable} out of a {\em row from a data frame} what we get is another, smaller, data frame:
\begin{lstlisting}
# Create a variable/data-frame  from a single row.
uk_data <- monPolData[1,]
uk_data
\end{lstlisting}
\begin{itemize}
	\item The first line of code creates the variable `ukData', the second prints it to the console. Which produces:
	\lstinputlisting[language=R, backgroundcolor=\color{lightgray}]{smallerDataFrame.out}
\end{itemize}
\newpage
	\item Finally on the subject of pulling data from a data frame, we can get hold of a single element.
\begin{lstlisting}
# Print just one element 
monPolData[1,3]
\end{lstlisting}
	\begin{itemize}
		\item Which shows up in the console as:
		\lstinputlisting[language=R, backgroundcolor=\color{lightgray}]{singleElement.out}
		\item I.e. from row 1 of the original data frame we get the data from column 3.
		\item We could of course create a variable out of this too but we don't do so here.
	\end{itemize}
\end{itemize}


\end{document}





# Print just 1 column
monPolData$bank

# Print just 1 row (you'll get the header too)

monPolData[1,]
monPolData[2,]

# Print just one element 
monPolData[2,3]

# How different do things look if we don't make the first line of the CSV file into a header in R?
----------
\begin{lstlisting}
#   Load data from csv with the top row of the csv used as the first row of the columns with column headers added by R (as V#).
monPolData_noHeader <- read.csv("monPol_1.csv",header=FALSE)
\end{lstlisting}
monPolData

monPolData_noHeader


# --------------- Load larger data set (from Radziwill book)

# Load data from csv with variable names as headers from the top row of the csv.
mnmData <- read.csv("mnm-clean.csv",header=TRUE)
mnmData_noHeader <- read.csv("mnm-clean.csv",header=FALSE)

# Look at the data...

#       The whole dataset
mnmData

#       Just the first 6 rows
head(mnmData)

#       What difference does including a 'header' make?
head(mnmData_noHeader)


# Print one of the columns
mnmData$student




\section{Earlier text}


\begin{enumerate}
		\item Import data from a spreadsheet or a `Comma Separated Values' (CSV) file :
		\begin{enumerate}
			\item See the \href{https://www.datacamp.com/community/tutorials/r-data-import-tutorial#csv}{Datacamp site} for a comprehensive guide.
			\item Spreadsheet \textcolor{red}{[mnm-clean.xlsx]}: Export the data as a CSV file \textcolor{red}{[mnm-clean.csv]}.  (File$$>$$Export$$>$$Change File Type$$>$$CSV (Comma delimited))
			\item CSV:  \textcolor{red}{[importDataFromCSV.R]}
		\end{enumerate}
		\begin{figure}[H]
			\begin{center}
  			\includegraphics[width=0.8\linewidth]{ScreenShots/Excel_mnmData.png}
  			\caption{Data in Excel.}
  			\label{fig:dataExcel}
  			\end{center}
		\end{figure}
		\begin{figure}[H]
			\begin{center}
  			\includegraphics[width=0.8\linewidth]{ScreenShots/R_mnmData.png}
  			\caption{Data in R.}
  			\label{fig:dataR}
  			\end{center}
		\end{figure}
		\item Data imported in this way are stored in R in something called a `data frame'; more on what this means below.
\end{enumerate}


\end{document}